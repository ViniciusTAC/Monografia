
% \index{CONCLUSÃO!exemplo de}
% \index{INTRODUÇÃO!conclusão amarrada com}
% A conclusão resume os principais pontos discutidos e apresenta as conclusões alcançadas a partir do trabalho.
% Ela destaca as descobertas mais significativas, sua relação com a literatura existente e suas implicações práticas ou teóricas.
% Além disso, a conclusão reafirma os objetivos do trabalho e sugere áreas para futuras investigações.
% É importante evitar a introdução de novas informações e manter a conclusão concisa e alinhada com os objetivos e resultados do estudo.

% Após a conclusão são apresentados alguns exemplos de elementos pós-textuais.
% Inclusive, elementos como apêndices e anexos devem ser referenciados.
% Como exemplo, exitem o Apêndice \ref{ap:exemplo} e o Anexo \ref{an:exemplo}.
