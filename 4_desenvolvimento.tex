Este capítulo apresenta o processo de desenvolvimento da ferramenta proposta para a coleta, estruturação e integração de dados financeiros públicos disponibilizados pela CVM, com foco em análises fundamentalistas de companhias abertas brasileiras.

As atividades foram organizadas em três frentes principais, que estruturam as seções deste capítulo. A Seção~\ref{sec:analise_cvm} descreve a análise preliminar dos dados disponibilizados pela CVM, com destaque para a identificação de padrões, inconsistências e limitações nos metadados. A Seção~\ref{sec:modelagem} detalha a modelagem e estruturação da base de dados, com ênfase na padronização relacional e na aplicação de boas práticas de modelagem para contextos financeiros, conforme discutido na literatura \cite{elmasri:2016:fundamentals}. Por fim, a Seção~\ref{sec:software} apresenta o sistema desenvolvido em Python, incluindo os módulos responsáveis pela coleta, transformação e armazenamento dos dados, bem como os mecanismos de registro de eventos (\textit{logs}).


\section{Análise inicial dos dados da CVM} \label{sec:analise_cvm}

A primeira etapa consistiu na compreensão do formato, da frequência de atualização e de outros aspectos relevantes relacionados aos dados disponibilizados pela CVM. Como o foco do trabalho são as Companhias (CIA) Abertas\footnote{\url{https://dados.cvm.gov.br}}, realizou-se uma análise do site como fonte para obtenção dessas informações. A partir disso, foram identificados os conjuntos de dados referentes às Companhias Abertas, organizados nas seguintes pastas:

\begin{itemize}
	\item informação cadastral;
	\item formulário cadastral;
	\item informações periódicas e eventuais;
	\item formulário de referência;
	\item valores mobiliários negociados e detidos;
	\item formulário de informações trimestrais;
	\item formulário de demonstrações financeiras padronizadas;
	\item informe do código de governança.
\end{itemize}

As informações cadastrais compõem uma categoria distinta, classificada como cadastro, enquanto os demais arquivos são agrupados sob a categoria de documentos. Ao acessar o site de dados da CVM\footnote{\url{https://dados.cvm.gov.br/dados/}}, especificamente a seção referente às Companhias Abertas, é possível visualizar essas duas categorias de dados.

Será apresentada, a seguir, a descrição de cada pasta referente ao conjunto de dados da CVM da CIA Aberta, com a finalidade de esclarecer o conteúdo e a utilidade de cada uma.

Adicionalmente, é importante contextualizar brevemente o sistema responsável pelo envio eletrônico dos documentos à CVM. O sistema \textit{Empresas.NET} (ENET) é a plataforma oficial utilizada pelas companhias abertas para a transmissão de informações regulatórias exigidas pela autarquia. Essa ferramenta garante a padronização, segurança e rastreabilidade do envio, sendo o meio exclusivo de entrega de documentos como o formulário de referência, as demonstrações financeiras padronizadas, os informes trimestrais e demais obrigações periódicas previstas nas normas vigentes.

Os dados disponibilizados no portal da CVM são oferecidos em formatos estruturados em \texttt{CSV} (Comma-Separated Values), em texto simples no formato \texttt{TXT} e arquivos comprimidos no formato \texttt{ZIP}. Esses formatos visam garantir maior acessibilidade e automação no tratamento das informações. A maioria dos conjuntos de dados está organizada em arquivos anuais, separados por tipo de formulário e data de referência, o que permite consultas retroativas e reprodutibilidade dos resultados.

Essas pastas listadas anteriormente podem conter dados referentes diretamente à categoria principal ou subdivisões internas em outras palavras subcategorias, conforme a complexidade do conteúdo. Os metadados associados a cada categoria são apresentados de duas formas: quando não há subcategorias, os metadados são oferecidos em arquivos de texto; caso contrário, são fornecidos em arquivos compactados que, ao serem descompactados, revelam arquivos textual para cada subcategoria.

Os dados propriamente ditos também são disponibilizados em arquivos compactados por ano. Após a extração, o conteúdo pode variar: em categorias sem subdivisões, o arquivo anual tende a conter um único documento estruturado; já em categorias com subcategorias, o arquivo anual inclui múltiplos arquivos estruturado, organizados conforme os diferentes tipos de informação tratados.

Dessa forma, observa-se um padrão de organização hierárquica: pastas por categoria principal, contendo arquivos de metadados e dados, com granularidade ajustada conforme a existência de subcategorias. Essa estrutura visa facilitar o acesso modular à informação alem de um controle sobre a atualização dos dados e permitir análises específicas com base no ano.

As legislações que fundamentam a disponibilização e a estrutura dos dados incluem, principalmente, a Resolução CVM nº 80/2022 \cite{cvm:2022:resolucao80}, que consolida as regras relativas ao registro e envio de informações pelas companhias abertas, e a Resolução CVM nº 44/2021 \cite{cvm:2021:resolucao44}, que dispõe sobre a divulgação de informações relativas a atos ou fatos relevantes, a negociação de valores mobiliários na pendência de tais informações ainda não divulgadas, bem como sobre a divulgação de operações realizadas com valores mobiliários por pessoas com acesso a informações privilegiadas. Tais normas conferem respaldo jurídico à padronização e à publicidade dos dados disponibilizados, sendo essenciais para o desenvolvimento de sistemas automatizados de análise fundamentada, como o proposto neste trabalho.

%Resolução cvm:2022:resolucao
%Resolução cvm:2021:resolucao44

\subsection{Informação cadastral} 

O conjunto Informação Cadastral reúne os dados cadastrais das companhias abertas disponibilizados pela CVM. Entre as informações disponibilizadas, destacam-se o número do Cadastro Nacional da Pessoa Jurídica (CNPJ), a data de registro da companhia e a situação atual desse registro. Esses dados são fundamentais para análises regulatórias, econômicas e financeiras, além de servirem como base para estudos acadêmicos e pesquisas de mercado. O dicionário de dados, disponibilizado em formato textual, contém a descrição detalhada das colunas e dos tipos de dados presentes no arquivo principal. Já os dados propriamente ditos, que contêm os registros cadastrais das companhias abertas, são apresentados em um arquivo estruturado.

Este conjunto de dados é de acesso público e está disponível na plataforma de dados abertos da CVM. A base é atualizada diariamente.

Para facilitar a compreensão geral da estrutura e do conteúdo disponibilizado, apresenta-se no Apêndice~\ref{ap:estrutura-informacao-cadastral} uma tabela-resumo com a visão geral dos campos e descrições desse conjunto de dados.

\subsection{Formulário cadastral}

O Formulário cadastral (FCA) é um documento eletrônico cuja entrega periódica ou eventual é regulamentada pelo a CVM e suas resoluções. Este formulário deve ser enviado à CVM por meio do sistema ENET, sendo uma obrigação regulatória destinada às companhias abertas. Sua principal função é atualizar e manter organizadas as informações institucionais e operacionais dessas entidades. O conjunto de dados públicos relacionados ao FCA, disponibilizado pela CVM, apresenta-se organizado em duas categorias principais de informações: os endereços de \textit{download} dos documentos completos submetidos pelas companhias abertas e os conteúdos estruturados das seções que compõem o formulário.

Seguindo o padrão já apresentando de subcategorias o FCA possui a categoria refere-se ao arquivo \texttt{fca\_cia\_aberta}, que contém os links para o \textit{download} dos documentos completos entregues pelas companhias nos últimos cinco anos. A visualização desses documentos requer a utilização do ENET. A as demais categoria compreende os conteúdos integrais do formulário, apresentados abaixo em arquivos estruturados por ano, estando dentro de um arquivo compactado por ano, conforme ja relatado anteriormente: 

\begin{figure}[!htb] \centering
	\caption{Estrutura do FCA} \label{fig:estrutura_fca}
	\begin{varwidth}{\linewidth}
		\dirtree{%
			.1 /.
			.2 fca\_cia\_aberta\_geral.
			.2 fca\_cia\_aberta\_pais\_estrangeiro\_negociacao.
			.2 fca\_cia\_aberta\_canal\_divulgacao.
			.2 fca\_cia\_aberta\_endereco.
			.2 fca\_cia\_aberta\_valor\_mobiliario.
			.2 fca\_cia\_aberta\_auditor.
			.2 fca\_cia\_aberta\_escriturador.
			.2 fca\_cia\_aberta\_dri.
			.2 fca\_cia\_aberta\_departamento\_acionistas.
		}
		\legend{Elaborado pelo autor, 2025.}
	\end{varwidth}
\end{figure}

Cada um desses arquivos aborda uma seção específica do formulário, possibilitando a segmentação e a análise detalhada das informações declaradas pelas companhias abertas Uma visão geral dos arquivos estruturados do FCA, com a respectiva correspondência normativa e observações de uso, está apresentada no Apêndice~\ref{ap:estrutura-fca}. Os arquivos são atualizados semanalmente, refletindo tanto novas entregas quanto reapresentações realizadas pelas companhias reguladas. Além disso, mantém-se um histórico contínuo desde o ano de 2010, incluindo inclusive arquivos que não estão sujeitos à política de atualização vigente. O dicionário de dados correspondente é disponibilizado em formato compactado e oferece descrições detalhadas de todas as colunas e variáveis presentes nos arquivos.

No desenvolvimento da ferramenta proposta neste trabalho, foi utilizado exclusivamente o arquivo \texttt{fca\_cia\_aberta\_geral}, por concentrar de forma consolidada as principais informações institucionais das companhias abertas, o que dispensa a necessidade de integrar os demais arquivos individualizados.

Para fins de análise temporal e organização, os formulários referentes aos anos de 2020 a 2025 encontram-se agrupados em arquivos anuais distintos. Este conjunto de dados integra a base denominada \textit{Documentos Periódicos e Eventuais de Regulados}. Sua manutenção ocorre de forma semanal, assegurando a atualização constante das informações disponibilizadas.

\subsection{Informações periódicas e eventuais}

O conjunto de dados referente às Informações Periódicas e Eventuais (IPE) reúne documentos não estruturados enviados por companhias abertas à CVM. Esses documentos, de caráter regulatório, contemplam tanto as obrigações periódicas quanto as comunicações eventuais exigidas ao longo das atividades societárias e da interlocução com o mercado. Trata-se de um acervo, que reflete a diversidade de exigências legais e normativas aplicáveis às companhias reguladas. O conteúdo do conjunto está organizado em seis grandes categorias documentais:

\begin{itemize}
	\item governança e estrutura societária;
	\item relação com investidores e mercado;
	\item informações econômico-financeiras e contábeis;
	\item transações e operações societárias;
	\item companhias em situação especial;
	\item informações regulatórias específicas.
\end{itemize}

Uma visão geral dos tipos documentais abrangidos pelo conjunto IPE pode ser consultada no Apêndice~\ref{ap:ipe-visao-geral}, que apresenta uma tabela que descreve melhor essas categorias nos tipos de documentos presente no IPE.

A categoria de governança e estrutura societária compreende documentos que regulam a organização interna e as diretrizes de conduta das companhias, como Acordos de Acionistas, Estatutos Sociais, Regimentos Internos, Códigos de Conduta, além de Políticas de Governança e Sustentabilidade. Esses registros são fundamentais para garantir a conformidade, a transparência e a integridade das práticas de gestão corporativa. 

Já os documentos de relação com investidores e mercado incluem comunicados oficiais direcionados ao público investidor e à sociedade, como Comunicados ao Mercado, Avisos a Acionistas e Debenturistas, Fatos Relevantes, Informações prestadas a bolsas estrangeiras e o Calendário de Eventos Corporativos. Tais documentos asseguram a divulgação tempestiva e adequada de informações relevantes, conforme exigido pelas normas da CVM.

No que se refere às informações econômico-financeiras e contábeis, o conjunto abrange dados sobre a situação patrimonial e os resultados das companhias, além de instrumentos financeiros e políticas de distribuição de lucros. Incluem-se nessa categoria Demonstrativos Financeiros, Documentos de Ofertas Públicas, Escrituras de Debêntures, Políticas de Dividendos e de Destinação de Resultados. A categoria de transações e operações societárias, por sua vez, contém registros de eventos relevantes que impactam a estrutura e o capital das empresas, como Comunicações sobre Transações com Partes Relacionadas, Ofertas Públicas de Aquisição (OPA) e Planos de Remuneração Baseados em Ações. 

Já a documentação relativa a companhias em situação especial reúne informações sobre empresas que enfrentam processos de falência, liquidação ou recuperação judicial/extrajudicial, sendo essenciais para análise de risco e monitoramento da situação jurídico-financeira dessas entidades. Por fim, a categoria de informações regulatórias específicas abrange documentos exigidos por normativos da CVM, como as comunicações sobre negociação de valores mobiliários por pessoas ligadas, políticas de atuação de auditores independentes e outras políticas internas obrigatórias.

Todos os documentos do conjunto IPE são disponibilizados em arquivos compactados organizados por ano, mesmo quando cada compactado contém apenas um único arquivo estruturado em seu interior. Os arquivos correspondentes ao ano corrente e ao ano imediatamente anterior são atualizados semanalmente, incorporando tanto novas submissões quanto eventuais reapresentações. O acervo possui cobertura histórica desde 2003, incluindo arquivos legados que, embora não estejam sujeitos à política atual de atualização contínua, permanecem acessíveis para consulta.

Cada conjunto anual acompanha um arquivo auxiliar contendo o dicionário de dados em formato textual, no qual são descritas as variáveis e a lógica de indexação aplicadas à organização dos documentos. A relevância dos arquivos IPE transcende o aspecto meramente documental: sua análise permite identificar mudanças estruturais relevantes, como alterações no nome empresarial, na área de atuação ou na composição societária, constituindo uma fonte estratégica para o monitoramento de eventos corporativos vinculados a um determinado \textit{ticker}.

\subsection{Formulário de Referência}

O Formulário de Referência (FRE) é um documento eletrônico cuja apresentação à CVM é obrigatória e periódica ou, em determinadas circunstâncias, eventual, por meio do sistema ENET. Sua principal finalidade é consolidar e divulgar, de maneira padronizada, um conjunto abrangente de informações sobre o emissor, contemplando sua estrutura societária, situação financeira, práticas de governança, riscos e relação com o mercado.

O conjunto de dados públicos associados ao FRE é composto por dois elementos principais:

\begin{itemize}
	\item os endereços para \textit{download} dos formulários submetidos;
	\item o conteúdo estruturado extraído desses documentos.
\end{itemize}

O primeiro item refere-se aos \textit{links} diretos para os formulários completos entregues pelas companhias abertas nos últimos cinco anos, disponibilizados principalmente por meio do arquivo \texttt{fre\_cia\_aberta}. Já o segundo item corresponde aos dados tabulares derivados do conteúdo dos formulários, organizados em arquivos estruturados que abordam diversas dimensões informacionais, agrupadas nas seguintes categorias:

\begin{itemize}
	\item informações institucionais e operacionais;
	\item aspectos financeiros e patrimoniais;
	\item administração e governança;
	\item ações e mercado;
	\item aspectos sociais e de diversidade.
\end{itemize}

Uma visão geral dos arquivos estruturados que compõem o conjunto FRE encontra-se apresentada no Apêndice~\ref{ap:fre-visao-geral}, com a descrição, referência normativa e indicação de uso no trabalho.

As informações estruturadas são atualizadas semanalmente, incorporando tanto novas submissões quanto reapresentações, e mantêm um histórico contínuo desde 2010. Essas atualizações derivam diretamente da base de Documentos Periódicos e Eventuais de Regulados submetidos à CVM. Cada pacote anual, disponível para os anos de 2010 a 2025, é disponibilizado em formato compactado e contém um conjunto completo de arquivos organizados por tipo de informação. Acompanha-se ainda um dicionário de dados, também compactado, que detalha as variáveis e colunas presentes.

Dentre os arquivos incluídos nesses pacotes, destaca-se o \texttt{fre\_cia\_aberta}, que possui especial relevância nesta pesquisa. Esse arquivo concentra, de forma consolidada, as principais informações estruturadas de cada submissão do FRE e inclui os identificadores e os links necessários para recuperação dos documentos completos no sistema ENET. Tal característica torna o \texttt{fre\_cia\_aberta} uma fonte centralizada e eficiente para acesso ao conteúdo essencial das companhias analisadas.

Embora os demais arquivos disponíveis nos pacotes — como \texttt{fre\_cia\_aberta\_remuneracao\_total\_orgao}, \texttt{fre\_cia\_aberta\_posicao\_acionaria}, entre outros, ofereçam informações relevantes e detalhadas sobre aspectos específicos do formulário, sua estrutura fragmentada e granular, aliada à sobreposição parcial de informações com o arquivo principal, levou à opção metodológica por sua não integração nesta etapa do trabalho. Assim, adotou-se exclusivamente o \texttt{fre\_cia\_aberta} como base de dados para análise, considerando sua abrangência e consistência com os objetivos da pesquisa.

Essa delimitação visa garantir maior simplicidade no modelo de integração dos dados, sem prejuízo da possibilidade de incorporação futura de arquivos complementares, conforme a evolução das necessidades analíticas.


\subsection{Valores mobiliários negociados e detidos}
O conjunto de dados Valores mobiliários negociados e detidos (VLMO) contempla informações de envio obrigatório à CVM por meio do ENET. Trata-se de uma obrigação de natureza periódica, cujo cumprimento deve ser realizado pelas companhias abertas. Esse conjunto tem como objetivo registrar e divulgar, de maneira transparente, a posição e as negociações realizadas com valores mobiliários por administradores, membros do conselho fiscal, controladores e pessoas a eles vinculadas. O conjunto disponibiliza os informes entregues pelas companhias nos últimos cinco anos, organizados em arquivos anuais compactados, contendo os dados estruturados. Esses arquivos reúnem informações como:

\begin{itemize}
	\item nome e CPF/CNPJ dos declarantes;
	\item tipos e quantidades de valores mobiliários detidos;
	\item natureza da operação (compra, venda, bonificação, entre outros);
	\item data e características das transações realizadas;
	\item relação do declarante com a companhia emissora.
\end{itemize}

Os arquivos são acompanhados de um dicionário de dados, também em formato compactado. O conjunto é atualizado semanalmente. O histórico apesar de informado pela CVM que abrange os anos de 2020 a 2025, não esta correto a informação pois é possível constatar dados de 2018. Ele utiliza como base os \textit{Documentos Periódicos e Eventuais de Regulados}.

Importa esclarecer que, para este estudo, não foi utilizado nenhum arquivo do conjunto VLMO. Destaca-se, ainda, que esse conjunto assim como os demais analisados apresenta uma estrutura padronizada, composta por dois arquivos principais: um arquivo nomeado como \textit{vlmo\_cia\_aberta}, que contém os endereços para \textit{download} dos documentos entregues pelas companhias, e outro arquivo estruturado que apresenta a consolidação dos dados do VLMO. Ambos os arquivos encontram-se organizados por ano e disponibilizados em formato compactado.

\subsection{Demonstrativos financeiros padronizados}
Os demonstrativos financeiros padronizados (DFP) disponibilizados pela CVM, reúnem dados contábeis estruturados de companhias abertas brasileiras. Tanto o ITR quanto o DFP seguem um modelo uniforme de reporte, que inclui as principais demonstrações financeiras, informações cadastrais, pareceres e arquivos para \textit{download}.

o conjunto de dados contempla as principais demonstrações financeiras, apresentadas em formato estruturado:

o conjunto de dados contempla as principais demonstrações financeiras, apresentadas em formato estruturado:

\begin{itemize}
	\item balanço patrimonial ativo (BPA);
	\item balanço patrimonial passivo (BPP);
	\item demonstrações dos fluxos de caixa - método direto (DFC-MD);
	\item demonstrações dos fluxos de caixa - método indireto (DFC-MI);
	\item demonstração das mutações do patrimônio líquido (DMPL);
	\item demonstração do resultado (DRE);
	\item demonstração do resultado abrangente (DRA);
	\item demonstração do valor adicionado (DVA).
\end{itemize}

%\begin{itemize}
%	\item balanço patrimonial ativo (BPA);
%	\item balanço patrimonial passivo (BPP);
%	\item demonstração dos fluxos de caixa (DFC), nas modalidades
%	\begin{itemize}
	%		\item método direto (DFC-MD);
	%		\item método indireto (DFC-MI);
	%	\end{itemize}
%	\item demonstração das mutações do patrimônio líquido (DMPL);
%	\item demonstrações do resultado
%	\begin{itemize}
	%		\item demonstração do resultado do exercício (DRE);
	%		\item demonstração do resultado abrangente (DRA);
	%	\end{itemize}
%	\item demonstração do valor adicionado (DVA).
%\end{itemize}



A principal distinção entre os dois documentos está na periodicidade, o ITR é divulgado trimestralmente, enquanto o DFP é publicado anualmente.

\subsubsection{Informações trimestrais}

As informações trimestrais (ITR) é um documento eletrônico de entrega obrigatória pelas companhias abertas à CVM. Ele reúne informações contábeis elaboradas de forma trimestral, em conformidade com as normas contábeis vigentes, com o objetivo de garantir a transparência e o acompanhamento contínuo do desempenho das empresas. Além das demonstrações financeiras, o ITR inclui declarações de responsáveis, dados cadastrais, composição do capital social e \textit{links} para \textit{download} dos arquivos completos. Os registros históricos remontam a 2011, e os dados são acompanhados por um dicionário de variáveis disponível em arquivos compactados.

\subsubsection{Formulário de demonstrações financeiras padronizadas}

O Formulário de demonstrações financeiras padronizada (DFP) reúne as demonstrações financeiras anuais das companhias abertas, conforme as resoluções apresentadas a cima, e é submetido exclusivamente por meio do sistema ENET. Esse formulário constitui uma base essencial para a análise da saúde patrimonial e econômica das empresas listadas no mercado de capitais brasileiro. O conjunto de dados é atualizado semanalmente e abrange informações desde 2010. Além das demonstrações financeiras, inclui pareceres de auditoria, declarações de administradores, dados cadastrais, composição acionária e \textit{links} para \textit{download} dos formulários. As demonstrações apresentam tanto contas fixas quanto variáveis, o que permite análises comparáveis e também personalizadas. Um dicionário técnico em formato compactado detalha todas as variáveis e campos disponíveis.

\subsection{Informe do Código de Governança}

O Informe do Código de Governança é um dos instrumentos regulatórios exigidos das companhias abertas brasileiras, com o objetivo de divulgar, de forma estruturada, as práticas de governança corporativa adotadas. Embora seja referenciado no site institucional da CVM pela sigla ICBGC, nos conjuntos de dados disponibilizados na plataforma de dados abertos, o conteúdo correspondente aparece sob a sigla CGVN. Considerando essa divergência de nomenclatura, este trabalho adota a sigla CGVN, conforme empregada nos arquivos estruturados.

O envio do CGVN é de caráter obrigatório e periódico, conforme estipulado pelas resoluções já apresentadas. O processo de entrega deve ser realizado exclusivamente por meio do sistema ENET, utilizado pelas companhias abertas para o cumprimento de suas obrigações junto à CVM.

Esse informe tem como propósito promover a transparência das práticas de governança corporativa, oferecendo à sociedade, aos investidores e aos demais agentes de mercado uma visão clara e comparável sobre como as companhias organizam sua estrutura decisória, implementam mecanismos de controle e estabelecem políticas institucionais. Também permite avaliar o grau de aderência aos princípios fundamentais de governança, como equidade, responsabilidade corporativa, prestação de contas e transparência.

Os principais tópicos abordados no CGVN são:

\begin{itemize}
	\item estrutura de governança vigente na companhia;
	\item composição e funcionamento dos órgãos de administração e fiscalização;
	\item políticas corporativas implementadas;
	\item nível de aderência às práticas recomendadas pelo Código Brasileiro de Governança Corporativa;
	\item justificativas apresentadas para os casos em que as recomendações não são seguidas.
\end{itemize}

O conteúdo informado pelas companhias é padronizado e estruturado para possibilitar análise automatizada em larga escala. Os dados abrangem o período de 2018 a 2025 e são atualizados semanalmente, contemplando tanto novas submissões quanto reapresentações.

O conjunto de dados CGVN é disponibilizado anualmente em arquivos compactados e organizados da seguinte forma:

\begin{itemize}
	\item \texttt{cgvm\_cia\_aberta};
	\item \texttt{cgvn\_cia\_aberta\_praticas}.
\end{itemize}

O primeiro arquivo contém os endereços para \textit{download} dos documentos completos entregues pelas companhias por meio do sistema ENET. Já o segundo arquivo apresenta os dados estruturados de forma tabular, organizados por companhia e prática declarada, permitindo processamento e análise automatizada.

O arquivo \texttt{cgvn\_cia\_aberta\_praticas} inclui colunas que indicam o CNPJ da companhia, a data de referência do informe, a versão do documento, o nome empresarial, os identificadores de documento e item, além dos campos descritivos com o capítulo do código, o princípio relacionado, a prática recomendada, a indicação de adoção e, quando aplicável, a justificativa para sua não adoção.

Cabe destacar que, para os fins deste trabalho, o conjunto de dados CGVN não foi incluído na análise prática, uma vez que o foco recai sobre conjuntos mais diretamente relacionados às demonstrações financeiras e ao cadastro das companhias.


\subsection{Resumo dos dados iniciais da CVM}

Com o intuito de compreender a estrutura e o comportamento dos dados disponibilizados pela CVM, foram desenvolvidos dois scripts em Python, descritos nos Apêndices~\ref{ap:codigo-baixar} e~\ref{ap:codigo-extrair}. O primeiro script realiza a varredura recursiva na pasta pública de dados abertos da CVM, efetuando o \textit{download} de todos os arquivos disponíveis sejam eles estruturados, compactados ou em formato textual sem a aplicação de filtros. O segundo script executa a extração em lote dos arquivos compactados, permitindo não apenas o acesso ao conteúdo interno, mas também a mensuração do volume, da diversidade e da granularidade dos dados presentes em cada conjunto. Essa etapa foi essencial para a análise preliminar, fornecendo uma visão concreta da organização e complexidade das bases tratadas.

Com base na análise exploratória viabilizada pelos \textit{scripts} desenvolvidos, foi possível identificar quais conjuntos de dados da CVM apresentam maior relevância, cobertura e estrutura adequada aos objetivos deste trabalho. A seleção final concentrou-se em bases que oferecem informações estruturadas, atualizadas e historicamente abrangentes, com potencial para análises fundamentalistas automatizadas. O Quadro~\ref{tab:comparativo_cvm} resume os principais aspectos dos conjuntos de dados escolhidos, os quais constituem o núcleo informacional utilizado ao longo deste estudo.

\begin{board}[!htb]
	\centering
	\caption{Comparativo entre conjuntos de dados da CVM}
	\label{tab:comparativo_cvm}
	\begin{varwidth}{\linewidth}
		\scriptsize
		\begin{tabularx}{\textwidth}{|X|X|X|X|X|X|}
			\hline
			\textbf{Aspecto}     & \textbf{ITR}                      & \textbf{DFP}                       & \textbf{FRE}                                    & \textbf{FCA}                       & \textbf{IPE}                                         \\ \hline
			
			Tipo de conteúdo    & Informações Trimestrais         & Demonstrações Financeiras Anuais & Formulário de Referência                      & Cadastro de Companhia Aberta       & Documentos Periódicos/Eventuais                     \\ \hline
			
			Frequência          & Trimestral                        & Anual                              & Periódico/Eventual                             & Periódico/Eventual                & Periódico/Eventual                                  \\ \hline
			
			Formato              & CSV em ZIP                        & CSV em ZIP                         & CSV em ZIP                                      & CSV em ZIP                         & CSV em ZIP                                           \\ \hline
			
			Volume descompactado & 9,62 GB                           & 3,49 GB                            & 941 MB                                          & 28,2 MB                            & 261 MB                                               \\ \hline
			
			Estrutura            & Demonstrativos contábeis         & Demonstrativos consolidados        & Informações qualitativas diversas             & Dados cadastrais padronizados      & Documentos PDF + metadados                           \\ \hline
			
			Atualização        & Semanal                           & Semanal                            & Semanal                                         & Semanal                            & Semanal (A e A-1)                                    \\ \hline
			
			Período histórico  & Desde 2011                        & Desde 2010                         & Desde 2010                                      & Desde 2010                         & Desde 2003                                           \\ \hline
			
			Destaques            & BPA, DRE, DFC, DMPL, DVA etc.     & BPA, DRE, DFC, DMPL, DVA etc.      & Capital social, remuneração, ESG, governança & CNPJ, endereço, auditor, canais   & Assembleias, fatos relevantes, estatutos, políticas \\ \hline
			
			Complexidade         & Alta (parse hierárquico)         & Média (consolidação)            & Alta (texto livre + variedade de temas)         & Baixa                              & Alta (texto livre, reapresentações)                \\ \hline
			
			Utilização         & Análise de desempenho trimestral & Análise patrimonial e histórica  & Análise estratégica e institucional           & Base de entidades (Dim\_Companhia) & Monitoramento de eventos corporativos                \\ \hline
		\end{tabularx}
		\legend{Elaborado pelo autor, 2025.}
	\end{varwidth}
\end{board}

A análise exploratória dos arquivos baixados e extraídos forneceu insumos valiosos para a etapa seguinte: o mapeamento dos dados. Essa fase consistiu na identificação das tabelas e colunas de origem, na avaliação de sua frequência de atualização e na estruturação de um modelo preliminar de destino. As informações coletadas por meio dos metadados auxiliaram na definição de domínios, tipos, tamanhos e descrições de campos, estabelecendo as bases técnicas para a modelagem relacional tratada na próxima seção.





\section{Modelagem e Estruturação dos Dados}\label{sec:modelagem}

A modelagem da base de dados teve início com o mapeamento das principais categorias disponibilizadas pela CVM — FCA, FRE, IPE, DFP e ITR — considerando também, quando existentes, suas respectivas subcategorias. Essa etapa preliminar teve como foco a análise da estrutura dos arquivos e dos metadados associados, permitindo compreender a organização, a granularidade e os padrões recorrentes nos dados fornecidos. O detalhamento desse processo encontra-se no Apêndice~\ref{ap:mapeamento-cvm-dfp}. O procedimento adotado seguiu uma lógica semelhante à de um processo ETL (\textit{Extract, Transform, Load}), com ênfase na padronização, organização e enriquecimento das informações extraídas das demonstrações financeiras reportadas por companhias abertas à CVM.

Durante a etapa de extração, foram utilizados arquivos textual que possuem os metadados, presente em arquivos compactados. Esses pacotes são atualizados semanalmente e organizados por tipo de demonstração contábil, como BPA, BPP, DRE, DFC-MD e DFC-MI, DMPL, DRA, DVA e pareceres de auditoria. Cada arquivo trata exclusivamente de um tipo de demonstração, garantindo segmentação e clareza na origem dos dados.

A etapa de transformação consistiu na padronização dos nomes de colunas e na normalização das estruturas tabulares. Realizou-se o mapeamento entre os campos originais e seus equivalentes no modelo relacional, adotando nomenclatura clara no padrão \texttt{snake\_case}. Exemplos incluem a substituição de \texttt{CD\_CVM} por \texttt{codigo\_cvm} e \texttt{DS\_CONTA} por \texttt{descricao\_conta}. Adicionalmente, foi realizado o enriquecimento temporal dos dados por meio da criação de três campos derivados a partir da data de referência (\texttt{DT\_REFER}): \texttt{data\_doc}, \texttt{mes\_doc} e \texttt{ano\_doc}. A tipagem dos campos foi definida com base em seus domínios originais, utilizando tipos como \texttt{varchar}, \texttt{decimal}, \texttt{date}, \texttt{smallint} e \texttt{char}, com destaque para colunas binárias com valores padronizados como “S” para sim ou “N” para não.

O carregamento dos dados estruturados ocorreu de forma centralizada, com a maioria das informações sendo direcionada à tabela \texttt{Dfp}, responsável por armazenar os demonstrativos financeiros. Casos específicos, como pareceres de auditoria, foram alocados na tabela \texttt{Dfp\_Parecer}. Adicionalmente, foi criada a tabela auxiliar \texttt{grupo\_dfp}, destinada a indicar se os dados referem-se a demonstrativos individuais ou consolidados. As atualizações seguem a mesma periodicidade dos arquivos-fonte, com carga semanal.

A estrutura resultante desse processo é padronizada e favorece a integração em repositórios analíticos, como \textit{data warehouses}, além de viabilizar consultas temporais e comparações entre companhias. Para isso, foram adotadas estratégias específicas de modelagem, como: 

\begin{itemize}
	\item redução de campos e tabelas desnecessárias;
	\item definição clara e consistente dos relacionamentos;
	\item unificação dos demonstrativos contábeis em tabelas únicas;
	\item criação de índices específicos para acelerar consultas analíticas.
\end{itemize}

Com base em reuniões periódicas com o orientador, definiu-se que o foco da modelagem seria a otimização da estrutura de dados com vistas à análise fundamentalista, priorizando desempenho, integridade e reprodutibilidade. A primeira versão do esquema foi desenvolvida no MySQL Workbench, conforme ilustrado no Apêndice~\ref{ap:esquema-inicial}. Essa versão inicial foi utilizada como base para testes de estruturação lógica e verificação do comportamento das tabelas, embora ainda apresentasse redundâncias e ausência de refinamentos.

Na sequência, foram realizados ajustes sucessivos no modelo, conforme apresentado no Apêndice~\ref{ap:esquema-algumas-modelagens}, com o objetivo de eliminar complexidades desnecessárias e tabelas pouco relevantes ao escopo do trabalho. A versão intermediária, descrita no Apêndice~\ref{ap:esquema-penultima-modelagem}, já incorporava a maior parte das estratégias descritas, refletindo uma estrutura mais coerente com a finalidade do projeto.

Por fim, a versão final do modelo lógico foi implementada em \texttt{SQLite} e encontra-se documentada no Apêndice~\ref{ap:esquema-logico-ultimo-presente-no-codigo}. Essa versão consolida as decisões de modelagem e incorpora tabelas auxiliares que contribuíram para a redução do tamanho da base e para o ganho de eficiência nas operações. Trata-se do modelo utilizado na ferramenta desenvolvida e adotado como base da integração de dados no sistema proposto.


\section{Características e Benefícios Analíticos da Estrutura Final}

A estrutura final, fruto de sucessivos refinamentos, possui as seguintes características principais:

\begin{itemize}
	\item Estrutura simples e clara;
	\item Facilidade de integração com sistemas externos;
	\item Disponibilidade pública e acessível.
\end{itemize}

A base contém apenas os dados essenciais, com nomenclaturas padronizadas e organização relacional intuitiva. Sua modularidade facilita a integração com outras fontes e ferramentas de análise.

Destaca-se ainda por ser aberta e pública, podendo ser utilizada em pesquisas acadêmicas, ferramentas de apoio à decisão financeira, além de iniciativas de transparência e educação financeira.

Entre os principais benefícios analíticos proporcionados pela base estão:

\begin{itemize}
	\item Facilidade na extração de indicadores financeiros;
	\item Avaliação precisa do endividamento;
	\item Estimativas confiáveis de valor intrínseco;
	\item Comparações históricas e setoriais consistentes;
	\item Apoio robusto à tomada de decisões financeiras.
\end{itemize}

Esses aspectos permitem análises fundamentalistas detalhadas, especialmente por meio da aplicação de indicadores de rentabilidade, como lucro por ação (LPA) e preço sobre lucro (P/L), além dos indicadores de liquidez, endividamento e das estimativas consistentes do valor intrínseco das empresas.

Por fim, a padronização dos dados possibilita comparações históricas e setoriais rigorosas, fornecendo um suporte consistente e confiável para investidores, pesquisadores e analistas financeiros.

\section{Software} \label{sec:software}





