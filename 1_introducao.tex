\index{INTRODUÇÃO}
O mercado financeiro brasileiro tem passado por transformações significativas impulsionadas pelo avanço tecnológico e pela crescente demanda por soluções que automatizem a integração e a análise de dados \cite{dantas:2020:comportamento}. Esse movimento reflete a busca constante por maior eficiência e acessibilidade nas informações financeiras, fundamentais para investidores e analistas que utilizam a análise fundamentalista como base para a tomada de decisões. 

O crescimento desse interesse pode ser observado no relatório anual de 2023 da Brasil, Bolsa e Balcão (B3), que apontou um aumento de aproximadamente 15\% no número de investidores em comparação com 2022 \cite{b3:2023:relatorio}. Esse cenário reforça a necessidade de garantir o acesso simplificado e estruturado às informações financeiras das empresas, promovendo maior eficiência nas análises e decisões de mercado.

No Brasil, dois órgãos desempenham papéis centrais na disponibilização de dados do mercado de capitais: a B3 e a Comissão de Valores Mobiliários (CVM). A B3, fundada em 1890 e sediada em São Paulo, é a única bolsa de valores do país e fornece dados sobre o histórico de negociações de ativos \cite{b3:2023:investidores}. 

Já a CVM, criada em 1976, é a entidade responsável por regulamentar e supervisionar o mercado de capitais brasileiro, disponibilizando publicamente uma ampla gama de informações contábeis e financeiras das empresas listadas na bolsa \cite{cvm:2009:informacao}. Essas informações são essenciais para investidores que utilizam a análise fundamentalista, metodologia que avalia a saúde financeira e o desempenho das empresas com base em seus demonstrativos financeiros e outros indicadores econômicos.

Embora a CVM disponibilize esses dados publicamente, seu formato dificulte a manipulação por usuários sem conhecimento técnico avançado. Muitos estão em arquivos complexos que exigem processamento adicional, criando um obstáculo para investidores e analistas que necessitam de informações ágeis e acessíveis. Essa barreira técnica restringe o acesso à informação e limita a capacidade analítica de boa parte do mercado.

A necessidade de soluções mais acessíveis já foi abordada por estudos acadêmicos. \cite{deAraujo:2021:modeloDados} propõem um modelo de dados flexível para análise fundamentalista moderna, estruturando balanço patrimonial, demonstração de resultados e fluxo de caixa em um banco de dados não relacional (MongoDB). Esse tipo de abordagem reforça a importância de alternativas que simplifiquem o acesso e a análise dos dados financeiros.


Diante desse contexto, este trabalho propõe o desenvolvimento de um sistema automatizado para a integração e análise dos dados fornecidos pela CVM, facilitando seu acesso e processamento. A solução busca oferecer uma ferramenta capaz de extrair e estruturar essas informações de forma eficiente, contribuindo para a análise fundamentalista e auxiliando na tomada de decisões. Além disso, pretende otimizar o processo de análise, tornando-o mais acessível e organizado para investidores, analistas e pesquisadores. Com essa abordagem, espera-se reduzir a complexidade no manuseio dos dados e aprimorar a compreensão do mercado.

\section{Justificativa}
A B3 apresentou um crescimento significativo no volume de negócios e de capital movimentado ao longo dos anos. Em 2023, o volume total negociado na B3 alcançou R\$ 7,2 trilhões, representando um aumento de 12,6\% em relação a 2022 \cite{b3:2023:investidores}. Além disso, o montante total de dinheiro movimentado na B3 em 2023 foi de R\$ 2,4 trilhões, um aumento de 15,5\% em relação ao ano anterior \cite{b3:2023:investidores}.

Embora os dados da B3 não estejam diretamente presentes neste trabalho, seu contexto é fundamental, uma vez que ela representa o principal ambiente de negociação de ativos no Brasil. As informações que embasam os negócios realizados na B3, especialmente aquelas relacionadas às demonstrações financeiras e dados cadastrais das empresas listadas, estão disponíveis na CVM. Dessa forma, a integração dos dados da CVM serve como uma base fundamental para análises que indiretamente impactam o entendimento do mercado da B3.

Atualmente, as informações da CVM estão disponíveis em formatos específicos, fragmentados e sem integração adequada, o que dificulta o acesso e a análise automatizada dos dados. Até o momento, não foram encontrados trabalhos que promovam essa integração de forma estruturada e com disponibilização pública.

A proposta de integrar os dados da CVM visa criar uma base de dados unificada e acessível, que sirva como suporte para futuros trabalhos acadêmicos e aplicações práticas em algoritmos de análise de mercado financeiro. Com isso, espera-se facilitar o desenvolvimento de estudos e ferramentas que contribuam para uma melhor compreensão do mercado de capitais brasileiro \cite{lindman:2020:integration}

\section{Objetivos}

Este estudo tem como objetivo principal o desenvolvimento de um sistema automatizado capaz de integrar e facilitar o acesso aos dados financeiros disponibilizados publicamente pela CVM. A proposta busca contribuir com a análise fundamentalista e apoiar o processo de tomada de decisão por parte de investidores, pesquisadores e demais interessados no mercado financeiro.

Para atingir esse objetivo, são estabelecidas as seguintes metas específicas:

\begin{itemize} 
	\item analisar a estrutura dos dados públicos fornecidos pela CVM, identificando suas fontes, formatos e padrões; 
	\item propor uma modelagem lógica que permita a integração eficiente e automatizada desses dados; 
	\item projetar e implementar um sistema que realize a extração, o processamento e a disponibilização dos dados de forma acessível e estruturada. 
\end{itemize}


\section{Resultados Esperados}

Espera-se que a ferramenta desenvolvida neste trabalho integre de forma eficiente os dados financeiros públicos disponibilizados pela CVM, permitindo a criação de uma base de dados unificada, estruturada e acessível.

Essa base será construída com foco em um recorte histórico que contemple, no mínimo, o período a partir de 2019, o que proporcionará maior profundidade nas análises e permitirá identificar padrões e variações ao longo do tempo. Tal histórico é essencial para estudos mais robustos e comparativos dentro do contexto da análise fundamentalista.

Embora os dados da B3 não sejam utilizados diretamente neste trabalho, o sistema proposto pode contribuir indiretamente para análises relacionadas à bolsa de valores, uma vez que as informações integradas da CVM servem como base para decisões tomadas por investidores e instituições que atuam no ambiente da B3.

A disponibilização pública dessa base de dados tem o potencial de promover maior transparência, acessibilidade e democratização das informações financeiras no Brasil, beneficiando pesquisadores, investidores e demais interessados no mercado de capitais.